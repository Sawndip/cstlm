%Language modelling is the task of estimating the probability of sequences of words in a language.
Language models (LMs) are one of the major and large components in many modern NLP systems, including machine translation XX and automatic speech recognition XX.
%
One approach for accurate language modeling is neural XX, but another is  count-based to harness the power of more data.
%
%The predominant approach to count-based language modelling is the $n$gram model,
%where a typical LM can contain as many as several hundred billions of $n$grams XX.
%
To be useful, LMs not only need to be accurate but they also need to be fast and compact.
%
%where the probability of a sequence of words is decomposed based on the chain rule ad approximatedusing the Markov assumption XX. 
%A typical count-based LM can contain as many as several hundred billions of $n$grams
In this paper, we build  fast and compact count-based language models on web-scale corpora using modern succinct data structures.
%
We show how to compactly index the text statistics needed by $n$gram language models, so that they can be queried efficiently to make the probabilities on the fly.       
%
Importantly, the stored index and text statistics are kept intact for $n$gram LMs with different orders (i.e. the context size), hence allow to relax the
Markov assumption made in these LMs by considering very large contexts.



Depending on the order $n$ and the training corpus size, a typical $n$gram LM may contain as many as several hundred billions of $n$grams [Brants et al 2007],
so storage and query time become challenge.
%
As always, there is a trade-off between accuracy, space, and time, with recent papers considering small but approximate  \emph{lossy} LMs [Chazelle et al., 2004;BRANTS],
or \emph{loss-less}  not-small LMs [SRI], where compression [Germann 2009], Suffix arrays [KenLM], suffix tries [KenLM], and suffix trees [LREC]  have been used to further reduce the storage.
% 
However, none of the aforementioned works scale up to high-order  $n$gram LMs for web scale corpora. 
%
We make use of recent advances in \emph{compact suffix arrays} (CSAs) [REF] and 
\emph{compact suffix trees} (CSTs) [REF] to build small-storage indices  with reasonable query time for high-order $n$gram LMs on web scale corpora.



%The predominant approach to count-based language modelling is the $n$gram model, where a typical LM can contain as many as several hundred billions of $n$grams XX.

  
%A typical LM can contain as many as severl hundred billions of $n$gramsFor modern NLP systems 
%The predominanr approach in language modelling is the $n$gram model, where the probability of a sequence of words is decomposed based on the chain rule and
%approximated using the Markov assumption XX. 
%
 

%The dominant approach to language modelling is $n$gram  model, where 

%LMs important, critical aspects are speed and scale. One avenue of better LMs using NN; another
%better count based LMs to harness more data (e.g., stupid backoff). 

%We're the latter, but marrying large scale with KN smoothing (better method). Crux is how to compute
%the quantities (counts, occ counts). 

%Suffix trees / arrays / tries etc used in the past. Either precomputed (KLM/SRI) or on the fly (slow
%LREC paper). But doesn't scale. We propose compressed SA/ST methods, which scale to massive datasets.

%Brief outline of how the method works; using one or two suffix trees. Forward/backward search.

%Pata on experiments: pplx good; can scale to large n and massive corpora; adequate runtime
%performance, around 100x slower than SRILM. Reranking gains? Can we run with INFINITE order?
